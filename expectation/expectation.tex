\documentclass{ximera}
\title{Expectation}
\outcome{Students will be able to calculate
and understand the meaning of the expected value of an experiment.}
\begin{document}
\begin{abstract}
This activity deals with the {\em expected value}
of an experiment.
\end{abstract}
\maketitle

This activity deals only with experiments
that have {\em numerical outcomes}.

\begin{example}
The following experiments have numerical outcomes.
\begin{itemize}
\item The number of minutes a randomly selected
customer waits in line at the grocery store
\item The number of olives in a randomly selected jar
\item The lifetime in hours of randomly selected light bulb
\end{itemize}
\end{example}

\begin{example}
The following experiments {\em do not} have numerical outcomes
\begin{itemize}
\item The result (heads or tails) of flipping a coin
\item The color of the shirt a randomly selected student is wearing
\end{itemize}
\end{example}

\begin{remark}
The outcomes of some experiments, while ostensibly numerical,
might fail to have significance as numbers.
For example, you could use a die to randomly select one of
six roommates to take the trash out.
In this case the outcomes $1,2,3,4,5,6$ correspond with people,
so the experiment is not considered to have numerical outcomes.
In contrast, in Monopoly the outcome of rolling two dice
determines the {\em number} of spaces a player advances.
In this situation the roll {\em is} considered to have a numerical outcome.
\end{remark}

\begin{exercise}
Do the following experiments have numerical outcomes?
\begin{itemize}
\item The number of books on a randomly selected
shelf at the library
\begin{multipleChoice}
\choice[correct]{Yes}
\choice{No}
\end{multipleChoice}
\item The first letter of the title of a randomly selected
book at the library
\begin{multipleChoice}
\choice{Yes}
\choice[correct]{No}
\end{multipleChoice}
\item The time at which the next book will be checked out
at the library
\begin{multipleChoice}
\choice{Yes}
\choice{No}
\choice[correct]{Possibly, depending on how the information will be used}
\end{multipleChoice}
\end{itemize}
\end{exercise}

Now suppose that an experiment has several outcomes
$E_1,E_2,\ldots,E_n$. These should all be numbers.
Also, suppose that $p_1$ is the probability of $E_1$,
$p_2$ is the probability of $E_2$, and so on.
Note that this means $1=p_1+p_2+\cdots+p_n$.

\begin{definition}
The {\em expectation} of the experiment described above
is defined to be
\begin{equation}\label{ExpectationDefinition}
p_1E_1+p_2E_2+\cdots+p_nE_n.
\end{equation}
\end{definition}

\begin{remark}
The reason we require the outcomes
$E_1,E_2,\ldots,E_n$ to be numbers is because
we add and multiply them in \autoref{ExpectationDefinition}.
This only makes sense when $E_1,E_2,\ldots,E_n$ are numbers.
\end{remark}

\begin{example}
Suppose you're taking an exam, but you
don't know the answer to one of the
multiple choice questions. You'll just have to guess or
else skip the question!
The question has five answer choices. You'll receive
$2$~points if you answer the question correctly,
but you'll {\em lose}
a half point for an incorrect response!
However, if you skip the question, you won't be penalized.
In other words, if you don't answer the question, your total
score on the exam will neither increase nor decrease.
Should you guess or skip the question?

We'll list all the outcomes with their probabilities,
assuming that you decide to guess.
\begin{itemize}
\item $2$ occurs with probability $\frac{1}{5}$ when
a correct answer is chosen
\item $-\frac{1}{2}$ occurs with probability $\frac{4}{5}$
when an incorrect answer is chosen
\end{itemize}
The expectation on the question is therefore
\[2\cdot\frac{1}{5}+\left(-\frac{1}{2}\right)\frac{4}{5}
=\frac{2}{5}-\frac{4}{10}
=\frac{4}{10}-\frac{4}{10}
=0.\]
So it doesn't hurt to guess!
However, see \autoref{SecondMultipleChoice} for a slight
variation on the scoring rules
with a very different conclusion.
\end{example}

\begin{remark}
Expectation gives
the expected value of an experiment {\em in long run}.
It doesn't tell us the exact value we expect
the next time the experiment repeated, because that
information would generally be impossible to know.
Expectation is used in the same way that averages are used.
\end{remark}

\end{document}
