\documentclass{ximera}
\title{Expectation}
\outcome{Students will be able to calculate
and understand the meaning of the expected value of an experiment.}
\begin{document}
\begin{abstract}
This activity deals with the {\em expected value}
of an experiment.
\end{abstract}
\maketitle

This activity deals only with experiments
that have {\em numerical outcomes}.

\begin{example}
The following experiments have numerical outcomes.
\begin{itemize}
\item The number of minutes a randomly selected
customer waits in line at the grocery store
\item The number of olives in a randomly selected jar
\item The lifetime in hours of randomly selected light bulb
\end{itemize}
\end{example}

\begin{example}
The following experiments {\em do not} have numerical outcomes
\begin{itemize}
\item The result (heads or tails) of flipping a coin
\item The color of the shirt a randomly selected student is wearing
\end{itemize}
\end{example}

\begin{remark}
The outcomes of some experiments, while ostensibly numerical,
might fail to have significance as numbers.
For example, you could use a die to randomly select one of
six roommates to take the trash out.
In this case the outcomes $1,2,3,4,5,6$ correspond with people,
so the experiment is not considered to have numerical outcomes.
In contrast, in Monopoly the outcome of rolling two dice
determines the {\em number} of spaces a player advances.
In this situation the roll {\em is} considered to have a numerical outcome.
\end{remark}

\end{document}
